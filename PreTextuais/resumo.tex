% The widespread adoption of the home office weakens corporate networks, as it extends its perimeter to homes and ineffective security policies designed for different operating environments. In this context, wireless network routers serve as enablers of access to critical services. This is why studying embedded devices' software security is important to help vendors identify software flaws that lead to security vulnerabilities so they can fix them and enhance the security of their devices. This way, a large-scale cybersecurity attack leveraging insecure IoT devices can be avoided. However, identifying the software artifacts and possible vulnerabilities present on these devices is challenging. 

% One approach for inspecting the security of embedded firmware is to use system emulation to re-host the firmware execution to another machine from which security analysis can be performed. Nonetheless, firmware re-hosting is an open problem, motivating a lot of popular research on new techniques and approaches. This work aims to explore wireless router firmware security by enumerating its content to leverage information and statistics that enhance the performance of state-of-the-art re-hosting solutions for firmware analysis.

% To achieve this, we present our efforts in the analysis of $9176$ firmware images downloaded from $11$ vendors' sites and $3$ open-source firmware projects, yielding statistics of the most common operating systems and services present on these devices and automatically generating reports containing the most relevant information and important exposed files found on each firmware. Afterward, we present our results when trying to apply state-of-the-art solutions of re-hosting to some of our acquired firmware images.

A adoção em larga escala do trabalho remoto enfraquece as redes corporativas, visto que o perímetro dessas redes é expandido para incluir domicílios e políticas ineficazes de segurança. Nesse contexto, roteadores de redes sem-fio servem como possibilitadores de acesso para serviços críticos. Por esse motivo, estudar a segurança de \textit{software} de dispositivos embarcados é importante para auxiliar fabricantes a encontrar falhas de \textit{software} que causem vulnerabilidades de segurança, para que estes consigam aplicar as devidas correções e aumentar a proteção de seus dispositivos.
Nesse sentido, pode-se evitar a ocorrência de um ataque cibernético em larga-escala que se aproveite de dispositivos \textit{IoT} inseguros. No entanto, identificar os artefatos de \textit{software} e possíveis vulnerabilidades de segurança presente nesses dispositivos é desafiador.

%Nesse sentido, um ataque cibernético em larga-escala se aproveitando de dispositivos de \textit{IoT} inseguros pode ser evitado. 

Uma abordagem para a inspeção de segurança de \textit{firmwares} embarcados é utilizar a emulação ao nível de sistema para realizar a execução do \textit{firmware} em outra máquina a partir da qual análises de segurança possam ser realizadas. No entanto, o \textit{re-hosting} de \textit{firmwares} é um problema em aberto, e motiva diversas pesquisas recentes sobre novas técnicas e abordagens. Este trabalho visa explorar a segurança de \textit{firmwares} de roteadores sem-fio através da enumeração de seu conteúdo para o levantamento de informações e estatísticas que melhorem o desempenho das soluções estado-da-arte para a análise de segurança de \textit{firmwares} via \textit{re-hosting}.

Para isso, serão apresentados os esforços realizados para a análise de $9.176$ arquivos de \textit{firmware} obtidos dos \textit{websites} de $11$ fabricantes e $3$ projetos de \textit{firmware} de código-aberto, produzindo estatísticas dos serviços e sistemas operacionais mais presentes nesses dispositivos, além da geração automática de relatórios contento as informações mais relevantes e arquivos expostos encontrados em cada \textit{firmware}. Posteriormente, serão também apresentados os resultados obtidos a partir da aplicação das ferramentas estado-da-arte para a execução de \textit{re-hosting}, nos arquivos de \textit{firmware} obtidos.