% ==========================================================
% The widespread adoption of the home office weakens corporate networks, as it extends its perimeter to homes and ineffective security policies designed for different operating environments. In this context, wireless network routers serve as enablers of access to critical services. However, identifying the software artifacts and possible vulnerabilities present on these devices is challenging, and a heuristic for this purpose is to obtain firmware available on the manufacturers' websites. This paper presents the analysis of $5265$ firmware images downloaded from $5$ vendors' sites, yielding a list of the most common operating systems and services, with the intent of, in a future work, perform security analysis in scale on the obtained firmware images. The exploitation of these components can lead to large-scale attacks, and our results contribute to the vulnerability cataloging process.
% ==========================================================
The widespread adoption of the home office weakens corporate networks, as it extends its perimeter to homes and ineffective security policies designed for different operating environments. In this context, wireless network routers serve as enablers of access to critical services. This is why studying embedded devices' software security is important to help vendors identify software flaws that lead to security vulnerabilities so they can fix them and enhance the security of their devices. This way, a large-scale cybersecurity attack leveraging insecure IoT devices can be avoided. However, identifying the software artifacts and possible vulnerabilities present on these devices is challenging. 

One approach for inspecting the security of embedded firmware is to use system emulation to re-host the firmware execution to another machine from which security analysis can be performed. Nonetheless, firmware re-hosting is an open problem, motivating a lot of popular research on new techniques and approaches. This work aims to explore wireless router firmware security by enumerating its content to leverage information and statistics that enhance the performance of state-of-the-art re-hosting solutions for firmware analysis.

To achieve this, we present our efforts in the analysis of 9,176 firmware images downloaded from 11 vendors' sites and 3 open-source firmware projects, yielding statistics of the most common operating systems and services present on these devices and automatically generating reports containing the most relevant information and important exposed files found on each firmware. Afterward, we present our results when applying state-of-the-art solutions of re-hosting to some of our acquired firmware images.

%, and a heuristic for this purpose is to obtain and inspect firmware available on manufacturers' websites.


%This paper presents the analysis of $5265$ firmware images downloaded from $5$ vendors' sites, yielding a list of the most common operating systems and services, with the intent of, in a future work, perform security analysis in scale on the obtained firmware images. The exploitation of these components can lead to large-scale attacks, and our results contribute to the vulnerability cataloging process.

% ==========================================================
% Studying embedded devices software security is important to help vendors identify software flaws that lead to security vulnerabilities so they can fix them and enhance their devices security. This way, a cybersecurity attack leveraging insecure IoT devices can be avoided. In this preliminary work we describe a way to automate security analysis on wireless routers firmware. Our proposal is to automatically acquire firmware images from vendor websites, extract kernel and filesystem from the acquired images and then re-host the firmware inside an emulator to use known techniques of vulnerability discovery.
% ==========================================================


% Studying embedded devices software security is important to help vendors identify software flaws that lead to security vulnerabilities so they can fix them and enhance their devices security. This way, a cybersecurity attack leveraging insecure IoT devices can be avoided. As wireless routers are usually the only line of defense between home users and small offices and the internet, assuring the security of these devices is very relevant. One way to determine if a firmware contain security vulnerabilities is to use system emulation to re-host firmware execution to another machine, from which security analysis can be performed. However, re-hosting firmware execution is very challenging, motivating a lot of popular research on new techniques and approaches.
