% In this chapter, we will present the architecture our research proposes in order to evaluate the large-scale re-hosting of router firmware images and the innovative aspects of the approach intended in this project. Hence, we define SCREEN: {\bf S}craper, {\bf C}lustering, {\bf RE}-hosting, and {\bf E}xploitatio{\bf N}.

In this chapter, we will present the architecture our research proposes in order to evaluate the large-scale re-hosting of router firmware images and the innovative aspects of the approach intended in this project. This paper represents only part of the solution idealized for a bigger research project, to be conducted throughout the years by a team of researchers that have already been contributing to our efforts. Firmware analysis is a new subject for most of the team, and so the work described in the paper is just the starting point of our research, in which we familiarize ourselves with the field and explore the state of the art related work. In this process we already envisioned research opportunities and possible paths that can expand the state of the art solutions in firmware re-hosting.

Our final goal with the complete research project is to evaluate the feasibility of a large scale network attack targeting SOHO wireless routers. As mentioned in Chapter \ref{chap:introduction}, we will be taking an offensive security approach, in which we will incorporate the attackers' role in trying to find configuration flaws and security breaches in firmware images that are operating in wireless router devices in the real world. To achieve that, we will need to acquire router firmware images and conduct our research on top of the acquired data. 

Hence, we define SCREEN: {\bf S}craper, {\bf C}lustering, {\bf RE}-hosting, and {\bf E}xploitatio{\bf N}, the solution whose architecture will be described next.

\section{Complete Architecture Idealized}

This paper represents only part of the solution idealized for a bigger research project, to be conducted throughout the years by a team of researchers. Figure \ref{fig:architecture} illustrates the overview of the SCREEN architecture for the complete project, whereas in this paper we will focus on the re-hosting part (highlighted on the Figure) of the proposed architecture. 

As the re-hosting requires already acquired firmware data, a scraper is also going to be required for the research to start. Because of that, we will take advantage of an already implemented scraper (firmware acquisition is described in Section \ref{sec:firmware-aquisition}), but exploring it's work or enhancing it's capabilities will be out of the scope proposed for this work.

\begin{figure}[h]
    \centering
    \includegraphics[width=0.85\textwidth]{figs/screen.pdf}
    \caption{Architecture proposed for SCREEN, our complete solution for firmware vulnerability analysis.}
    \label{fig:architecture}
\end{figure}


Considering the complete architecture illustrated by Figure \ref{fig:architecture}, everything starts with the scraper module (just mentioned) consisting of a web crawler responsible for entering router vendors websites and downloading for a local repository (in our case to a simple directory) the biggest amount of firmware images available as possible. As mentioned, firmware images are a requirement for the re-hosting experiments to be conducted. So, in this stage, the scraper module is going to only serve the purpose of providing a minimum amount of firmware images to allow the research to proceed. Therefore, our approach will be to use a fork of the original Firmadyne's \cite{firmadyne} implemented scraper for now. In the future, if needed, our team is willing to update Firmadyne's scraper and also add more vendors to the list.

% From the local repository, two different modules will perform actions using the acquired firmware. The re-hosting module will be responsible for extracting firmware images (and collecting information about the firmware in the process) and preparing the firmware image to be emulated. As this module will be the focus of this research, section \ref{sec:re-hosting} will explain this process in detail. The other module to perform actions in firmware images contained in the local repository is the clustering module. This consists on searching for similarities and patterns amongst different firmware images, and collecting the most common artifacts present on the firmware images. The results obtained by the clustering modules can be used in further steps to improve the vulnerability analysis and discovery process.

From the local repository, two different modules will perform actions using the acquired firmware. The re-hosting module (main subject of this work) will be responsible for extracting firmware images (and collecting information about the firmware in the process), enumerating firmware content, extracting statistics from the firmware repository and for preparing the firmware image to be emulated. As this module will be the focus of this research, Sections \ref{sec:extraction} and \ref{sec:re-hosting} will explain this process in detail. For now, let us continue explaining the proposed architecture. 

% The other module to perform actions in firmware images contained in the local repository is the clustering module. This consists on searching for similarities and patterns amongst different firmware images, and collecting the most common artifacts present on the firmware images. The results obtained by the clustering modules can be used in further steps to improve the vulnerability analysis and discovery process.

The other module to perform actions in firmware images contained in the local repository is the clustering module. This consists on searching for similarities and patterns amongst different firmware images, and collecting the most common artifacts present on the firmware images. The idea is to gather a lot of data in this process and try to apply state of art pattern discovery techniques. The plan is that the results obtained by the clustering module could be used in further steps to improve the vulnerability analysis and discovery process. For now this module is only in the idealized architecture and this work does not contain any advances in implementing the clustering module. In fact, the clustering and exploitation modules are still subjects to be designed in further steps of the research to be conducted by our SCREEN contributors.

The final module of the architecture is the exploitation module, that as just mentioned is still only an idealization and open to changes. In this stage, the plan is to apply vulnerability analysis (using frameworks to detect known software vulnerabilities - such as the Metasploit framework) and vulnerability discovery (such as fuzzing techniques) on the emulated firmware to analyze it's security performance.

In the next subsections we will describe more about how the extraction, enumeration and re-hosting of the firmware files are going to be accomplished.

\section{Firmware Extraction}
\label{sec:extraction}

For the firmware extraction process, our research is going to focus on enhancing the original script provided with Firmadyne \cite{firmadyne} for firmware extraction developed in Python and making extensive use of {\tt binwalk}'s API. This original script uses {\tt binwalk} to recursively extract the firmware image setting a limit for depth and breadth in order to limit this process as it is very slow. To speed up the extraction, the {\tt /tmp} directory of the host system (which is used to temporarily store files during the firmware extraction process) is going to be mounted on the {\tt tmpfs} provided by the Linux kernel and that allow us to mount a directory in the Random Access Memory (RAM) instead of mounting it on the disk.

This can be done per session, using the {\tt mount} command, or an entry can be added to the {\tt /etc/fstab} file of the Linux system. The entries found in {\tt /etc/fstab} will be automatically mounted during the system boot process. For instance, for manually mounting a directory named {\tt /home/firm/extractdir} with 16GB of size into the {\tt tmpfs} (RAM) for a single session, command show on Code \ref{code:tmpfs-manual} may be used (requires elevated privileges to run).

\begin{listing}[H]
\inputminted[breaklines]{text}{Code/tmpfs-mount}
\caption{Command line to mount a directory into the TMPFS (RAM memory). Must be executed with root privileges.}
\label{code:tmpfs-manual}
\end{listing}

For automatically mounting the same directory into the {\tt tmpfs} during the boot process, with a size of 16 GB, the entry shown on Code \ref{code:tmpfs-fstab} should be added to the {\tt /etc/fstab} file:

\begin{listing}[H]
\inputminted[breaklines]{text}{Code/tmpfs-fstab}
\caption{Entry that need to be added to the {\tt /etc/fstab} file in order to automatically mount a directory into the TMPFS during the boot process.}
\label{code:tmpfs-fstab}
\end{listing}

Nonetheless, this method of firmware extraction using the customized Firmadyne's \cite{firmadyne} extraction script is potentially incapable of automatically extract some firmware images. Because not all firmware follows standard implementation, and frequently vendors obfuscate compression and file systems, we marked the failure subjects to posterior examination. Afterward, we inspect the marked ones to identify constraints and verify how to improve the automated image extraction. Then, subsequent modifying the extraction script, this entire process is repeated until satisfactory reach the maximum level in the extracted firmware images.

\section{Firmware Enumeration}

During the extraction phase of a firmware, the extraction script, atop of Binwalk, tried to infer information about the kernel version within the firmware and the original architecture of the firmware in question. If these pieces of information are obtainable, they are stored within a database, associating the information with a firmware id (each firmware receives an unique identification number).

After a firmware filesystem is successfully extracted using the aforementioned extraction script, it's content is then compressed and stored in a specific directory. Firmadyne already implements one tool that helps the task of enumerating firmware content. A Python script (named {\tt tar2db.py}) can be used to list the contents of a compressed filesystem for a given firmware image, and then the name of every file listed is stored in a database, associating the filename with the original firmware file.

Leveraging these mechanisms, we will explore on how to automatically detect and extract data from within the firmware images together with a way to store and visualize this data.

\section{Re-hosting Process}
\label{sec:re-hosting}

We aim to maximize the portion of the original firmware during the re-hosting phase, absenting the original hardware. Firmadyne's approach to firmware re-hosting extracts from the original firmware image its original root filesystem and kernel. It then completely ignores the original kernel and replaces it with a custom instrumented kernel designed by the researchers (they developed one instrumented kernel for {\tt ARM} architecture and one for {\tt MIPS} architecture). Finally, the {\tt QEMU} tool is responsible for the emulation process, which uses binary translation to allow the execution of binaries from different architectures. The emulated firmware is emulated with {\tt QEMU} using the instrumented kernel together with the extracted filesystem from the original firmware.

Our approach to firmware re-hosting is similar to Firmadyne's one, with differences regarding the kernel replacement part. Instead of just building one heavily instrumented kernel and replacing all firmware images with that same instrumented kernel, we propose building-specific kernel versions on demand according to the original kernel used by the firmware image under emulation. With that, our heuristics correspond to use a more similar kernel to the original to reduce incompatibility with network drivers and modules and therefore increase the number of emulated firmware with a working network interface. In addition, increasing the emulation coverage allow us to extend the vulnerability analysis and discovery process to a more significant amount of router firmware images than the one covered by the original Firmadyne's implementation.

\subsection{Kernel Automated Compile}

This process of automated kernel cross-compilation, however, faces many issues. The first one is how to define a valid kernel compile configuration that results in a kernel that has the features expected in router firmware. Kernel compile configuration is a configuration file ({\tt .config}) where the user can configure an extensive list of parameters to include or exclude features to the compiled kernel.

These configuration files can be filled up by the user manually, via answering questions interactively in the command-line interface, or ultimately adopting an existing file used by a current firmware's kernel previously compiled. Defining how to produce a good {\tt .config} file is already an open problem in our research. Another idea is to extract kernel compilation default configuration from OpenWrt Linux systems. The OpenWrt is a Linux targeted to serve as an open-firmware for wireless routers. Thereby, OpenWrt images may have kernel configurations compatible with most networking features expected from a wireless router. In this way, in this research, we plan to investigate the better way to produce a valid compilation configuration file to allow kernel compiling.

After choosing a valid kernel compilation configuration file, the kernel has to be cross-compiled in the host machine to compile for the architecture expected in the original firmware image. Unfortunately, kernel cross-compilation is also challenging, as the compilation process is heavily dependent on its toolchain (compiler, utilities, and libraries versions). Furthermore, it means that to compile multiple different kernel versions automatically; there must be a way to automatically switch between a set of known working toolchains for each kernel version.

Another approach, to reduce the number of parameters that need to be configured in order to build a very tailored kernel for each firmware, could be to use the enumeration and focus on building a repository with the most common kernel versions (or at least kernel families). Then, for the re-hosting, for each firmware image, an heuristic could be used to determine which compiled kernel from the repository is the most similar to the original kernel expected from the firmware. This matching kernel could then be selected to be used during the re-hosting process.

After compiling a kernel version that matches the original kernel, the firmware image is then ready to be emulated using the {\tt QEMU}. First, the newly compiled kernel serves as firmware in emulation with the original filesystem extracted from the firmware sample. After that, following the work of \cite{firmadyne}, we evaluate if these emulated firmware images can infer and configure the network successfully. If that is the case, vulnerability analysis and discovery techniques are applied (using standard tools for this purpose). Finally, the results obtained are helpful to evaluate overall firmware security, reporting the mapped vulnerable products (if that happens) to the vendor.

Figure \ref{fig:firmadyne-screen-compare} illustrates how the SCREEN approach to re-hosting differs from the one used by Firmadyne. While Firmadyne uses one heavily instrumented kernel with all firmware images during the re-hosting process, SCREEN will build kernels using the information gathered from the firmware images and select a more suitable kernel from the repository to use when re-hosting a specific firmware image.

\begin{figure}[h]
     \centering
     \begin{subfigure}[b]{0.45\textwidth}
         \centering
         \includegraphics[width=\textwidth]{figs/Firmadyne-Approach.pdf}
         \caption{Firmadyne approach to re-hosting.}
         \label{fig:firmadyne-approach}
     \end{subfigure}
     \hfill
     \begin{subfigure}[b]{0.45\textwidth}
         \centering
         \includegraphics[width=\textwidth]{figs/SCREEN-Approach.pdf}
         \caption{SCREEN approach to re-hosting.}
         \label{fig:screen-approach}
     \end{subfigure}
        \caption{Comparison between the approaches used by Firmadyne and SCREEN during the re-hosting phase.}
        \label{fig:firmadyne-screen-compare}
\end{figure}

\subsection{Re-hosting execution}

We will also explore how to manually perform a firmware re-hosting by using {\tt QEMU} to start a system mode emulation with a minimal filesystem and kernel and then copy to this minimal system the essential files and configuration responsible for executing the basic behavior expected from the firmware. This will be done as a way to gain a deeper understanding on how to use the {\tt QEMU} emulator and to gain knowledge about the practical aspects of emulating a firmware.

After that, we will explore how to perform re-hosting using other research products such as Firmadyne~\cite{firmadyne} and Jetset~\cite{jetset} with a critical view on how these solutions could be modified to enhance the re-hosting phase and increase the number of successfully re-hosted firmware, as this would allow us to (in further research projects) to perform security analysis on a larger number of firmware images.