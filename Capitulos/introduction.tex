In the last decades, the ability to work in a person's home has been an increasing desire in society, and following the fast advances in technology, companies were already experimenting with remote models of work. Amidst the COVID-19 pandemic, many cities imposed mobility restrictions in order to restrain virus spreading.  Henceforth, companies have adopted remote work, and there is a tendency to increase this model considerably in a pos-covid world.

Working from home expands the companies' network perimeter, exposing digital assets to new threats and vulnerabilities. Consequently, it causes an increase in a company's attack surface as small and home office types of equipment are potentially more vulnerable. Home wireless routers, for instance, are the worker's first access to the internet and maybe running firmware with security breaches that could leverage to provoke a cybersecurity incident.

One approach is to detect vulnerabilities in firmware products before the attackers and report them back to the vendor to prevent this kind of attack. Thereby, the manufactures can patch the system to fix the identified security breaches. This paper aims to discuss a way to automate the security analysis and vulnerability detection in wireless router firmware via re-hosting the original firmware in an emulator before executing vulnerability analysis and discovery techniques.